\documentclass{beamer}
\usepackage{listings}
\usepackage{beamerthemesplit}
\usepackage{hyperref}

\title{Test reporting with Coverage}
\author{Dan Colish}
\date{\today}

\begin{document}
\frame{\titlepage}

\frame
{
  \frametitle{Checking your code coverage!}
  \begin{center}
  \includegraphics[scale=0.35]{laughing-goat.jpeg}
  \end{center}
  \begin{itemize}
    \item Shows the effectivness of your testing
    \item Highlights areas for new or increased testing
    \item Helps to verify that our tests do what we think they do
  \end{itemize}
  
  The utility for test coverage is coverage.py \\
  \small\url{http://nedbatchelder.com/code/coverage/}.
}

\frame {
  \frametitle{Using coverage.py}
  \begin{itemize}
   \item Features
     \begin{itemize}
     \item Integrates with the nose test suite 
     \item Generates a variety of HTML, XML, and console reports
     \item Can be called from anywhere in your code
     \item Tracks multiple subprocesses and mutliple code branches
     \end{itemize}
   \item Configuration
     \begin{itemize}
       \item Commandline options ('--omit', '--branch', etc)
       \item .coveragerc [this is better]
       \end{itemize}

   \item Running It
     \begin{itemize}
       \item With nose use '--with-coverage --cover-package=[some
         package]'
       \item On the commandline 'coverage run [options]'
     \end{itemize}
   \end{itemize}
}

\begin{frame}[fragile]
  \frametitle{We need to talk about your TPS reports.}
  If you ran your tests with `-p` or in parallel, you'll need to
  execute `coverage combine` first.
  \begin{itemize}
    \item HTML Reporting
      \begin{itemize} 
        \item Generate with `coverage html`
        \item use '-d' to specify a directory for your output
        \item open index.html in that dir for reports
      \end{itemize}
   \item XML Reporting
     \begin{itemize}
       \item Generate with `coverage xml`
       \item similar options to html reports
       \item integrates pretty well with Hudson
     \end{itemize}
  \end{itemize}
  You will need to omit things like Genshi templates report
\lstset{language=Python,basicstyle=\tiny}
\begin{lstlisting}
[html]
directory=coverage_html_report
omit=lib,tests2,idealist/webapp/templates/,idealist/webapp/org/templates/
\end{lstlisting}
\end{frame}  


\end{document}
   