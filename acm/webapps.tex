\documentclass{beamer}
\usepackage{amsmath}
\usepackage{listings}
\usepackage{beamerthemeCopenhagen}
\usepackage{hyperref}
\usepackage{graphviz}


\title{Web Programming}
\author{Dan Colish}
\date{\today}

\begin{document}
\frame{\titlepage}
\frame[<+->]
{
  \frametitle{WTF is web programming}
  \begin{block}{What its not}
    \begin{itemize}[<1->]
    \item trivial or easy
    \item not worth proper engineering
    \item not CS enough
    \item LOLcats
   \end{itemize}
  \end{block}

  \begin{block}{What it is}
    \begin{itemize}[<1->]
    \item Highly sophisticated
    \item deals with core issues of CS
    \item easy to get wrong
    \item difficult to avoid
    \end{itemize}
  \end{block}
}

\frame[<+->]
{
  \frametitle{Anatomy of a webapp}
  \begin{block}{Routing and handling}
    \begin{itemize}[<+->]
    \item Webapps typically consist of two main components
      \begin{itemize}[<+->]
      \item Router: matches url endpoints to handlers
      \item Handlers: function to execute with incoming request data
      \end{itemize}
    \item This is a very common pattern for event driven programming!
    \item Routing logically maps requests to the objects it returns
      
    \end{itemize}
  \end{block}
}

\frame[<+->]{
  \frametitle{ZOMG THE BIG PICTURE!}
  \begin{block}{The server request/response cycle}
    \begin{itemize}[<+->]
    \item server receives request from the network
    \item passes endpoint through the router
    \item router matches to handler and dispatches(calls handler)
    \item handler returns repsonse to client
    \end{itemize}
  \end{block}

  \begin{block}{What does that mean?}
    \begin{itemize}[<+->]
    \item objects typically don't persist a request/response cycle
    \item you'll need to store the data elsewhere, then load per request
    \end{itemize}
  \end{block}
}

\frame{
  \frametitle{whats it look like}
  \begin{block}{HelloWorld}
    A simple application pseudocode might look like this
    \lstinputlisting[basicstyle=\tiny]{pseudocode.txt}
  \end{block}
}

\frame{
  \frametitle{did that just happen?}
  \begin{block}{breaking it down, one more time}
  \begin{itemize} 
  \item application waits for a call to the router; typically the main entrypoint
  \item router matches url pattern and passes to mapped function
  \item function executes and returns response data
  \end{itemize}
  \end{block}

  \bf{OK, so whats so hard about that?}
}

\frame[<+->]
{
  \frametitle{The devil is in the details}
  \begin{block}{Mistakes we've made}
    \begin{itemize}
    \item too many different url/view classes
    \item you didnt think about your data model before coding
    \item your forms don't validate
    \item the code base is a mess of circular imports
    \end{itemize}
  \end{block}

  \begin{block}{What to do about it}
    \begin{itemize}
      \item keep your map simple and use smarter matching
      \item really consider what data you'll need and how you're going to manage it
      \item ***HOLY COW! ALWAYS VALIDATE FORM DATA***
      \item keep you code clean and well organized, take time to have reviews
      \item scaling isn't about adding users, its about adding code
      \end{itemize}
    \end{block}
}

\frame
{
  \frametitle{Questions??}
  \begin{block}{Contact Info}
  \begin{itemize}
    \item Dan Colish [dcolish@gmail.com]
    \item http://mostly-decidable.org
  \end{itemize}
\end{block}
}

\end{document}