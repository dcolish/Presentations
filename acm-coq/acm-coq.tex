\documentclass{beamer}
\usepackage{beamerthemesplit}
\usepackage{coqdoc}

\title{Coq: it has an unfortunate name}
\author{Dan Colish}
\date{\today}

\begin{document}
\frame{\titlepage}

\section{Intro}
\subsection{What is Coq?}
\frame
{
  \frametitle{Why Prove anything at all?}

  \begin{itemize}
    \item Formally state mathematical theories
    \item Define predicate functions that can be evaluated systematically
    \item Verifies that a program does what we expect it to.
  \end{itemize}
}

\frame {
  \frametitle{Computer Checking Proofs}
  \begin{itemize}
    \item Hand-written proofs are subject to mistakes
    \item Considered to be informal since there is no formal method of
      checking
    \item Coq allows us to systematically verify proofs using the
      Calculus of Inductive Constructions
    \item All decisions about Coq proofs are resolved using very
      strong type checking
   \end{itemize}
}

\frame {
  \frametitle{So What's it look like?}
    Here's something simple.

  \begin{coqdoccode}
\scriptsize{   \coqdockw{Theorem} \coqlemma{acm.sillyprop}{silly\_prop} :forall (\coqdocvar{A} \coqdocvar{B} \coqdocvar{C}: \coqdockw{Prop}), \coqdoceol
\coqdocindent{1.00em}
(\coqdocvar{A} \ensuremath{\rightarrow} (\coqdocvar{B} \ensuremath{\rightarrow} \coqdocvar{C})) \ensuremath{\rightarrow} (\coqdocvar{A} \ensuremath{\rightarrow} \coqdocvar{B}) \ensuremath{\rightarrow} (\coqdocvar{A} \ensuremath{\rightarrow} \coqdocvar{C}).\coqdoceol
\coqdocnoindent
\coqdockw{Proof}.\coqdoceol
\coqdocindent{1.00em}
\coqdoctac{intros} \coqdocvar{A} \coqdocvar{B} \coqdocvar{C} \coqdocvar{H1} \coqdocvar{HA} \coqdocvar{H2}.\coqdoceol
\coqdocindent{1.00em}
\coqdoctac{apply} \coqdocvar{H1}.\coqdoceol
\coqdocindent{1.00em}
\coqdoctac{apply} \coqdocvar{H2}.\coqdoceol
\coqdocindent{1.00em}
\coqdoctac{apply} \coqdocvar{HA}.\coqdoceol
\coqdocindent{1.00em}
\coqdoctac{assumption}.\coqdoceol
\coqdocnoindent
\coqdockw{Qed}.\coqdoceol
}\end{coqdoccode}

And a little more complex...

\begin{coqdoccode}
\scriptsize{
\coqdocnoindent
\coqdockw{Theorem} \coqdocvar{ex1} : \ensuremath{\forall} (\coqdocvar{A} \coqdocvar{B} \coqdocvar{C} \coqdocvar{D}:Prop),\coqdoceol
\coqdocindent{1.00em}
(\coqdocvar{A} \ensuremath{\lor} \coqdocvar{B}) \ensuremath{\rightarrow} (\coqdocvar{B} \ensuremath{\land} \coqdocvar{C}) \ensuremath{\rightarrow} (\coqdocvar{B} \ensuremath{\rightarrow} \coqdocvar{C}) \ensuremath{\lor} \coqdocvar{D}.\coqdoceol
\coqdocnoindent
\coqdockw{Proof}.\coqdoceol
\coqdocnoindent
\coqdoctac{intros}.\coqdoceol
\coqdocnoindent
\coqdoctac{destruct} \coqdocvar{H0}.\coqdoceol
\coqdocnoindent
\coqdocvar{left}.\coqdoceol
\coqdocnoindent
\coqdoctac{intros}.\coqdoceol
\coqdocnoindent
\coqdoctac{apply} \coqdocvar{H1}.\coqdoceol
\coqdocnoindent
\coqdockw{Qed}.\coqdoceol
}\end{coqdoccode}  
}

\frame {
  \frametitle{That's cool, now what?}
  Now we can start reasoning about programs. Lets say we have the following functions on lists.
  
First we need to tell Coq how to construct a list \begin{coqdoccode}
\scriptsize{
\coqdocnoindent
\coqdockw{Inductive} \coqinductive{simplelists.natlist}{natlist} : \coqdockw{Type} :=\coqdoceol
\coqdocindent{1.00em}
\ensuremath{|} \coqconstructor{simplelists.nil}{nil} : \coqinductiveref{simplelists.natlist}{natlist}\coqdoceol
\coqdocindent{1.00em}
\ensuremath{|} \coqconstructor{simplelists.cons}{cons} : \coqinductiveref{Coq.Init.Datatypes.nat}{nat} \ensuremath{\rightarrow} \coqinductiveref{simplelists.natlist}{natlist} \ensuremath{\rightarrow} \coqinductiveref{simplelists.natlist}{natlist}.\coqdoceol
\coqdocemptyline
}\end{coqdoccode}
Using this constructor we can then define lists \begin{coqdoccode} \scriptsize{
\coqdocemptyline
\coqdocnoindent
\coqdockw{Definition} \coqdefinition{simplelists.l123}{l\_123} := \coqconstructorref{simplelists.cons}{cons} 1 (\coqconstructorref{simplelists.cons}{cons} 2 (\coqconstructorref{simplelists.cons}{cons} 3 \coqconstructorref{simplelists.nil}{nil})).\coqdoceol
\coqdocemptyline
}\end{coqdoccode}
Is is trivial to add notation for easier use of the list constructor \begin{coqdoccode}
\scriptsize {
\coqdocnoindent
\coqdockw{Notation} "x :: l" := (\coqconstructorref{simplelists.cons}{cons} \coqdocvar{x} \coqdocvar{l}) (\coqdoctac{at} \coqdocvar{level} 60, \coqdocvar{right} \coqdocvar{associativity}).\coqdoceol
\coqdocnoindent
\coqdockw{Notation} "[ ]" := \coqconstructorref{simplelists.nil}{nil}.\coqdoceol
\coqdocnoindent
\coqdockw{Notation} "[ x , .. , y ]" := (\coqconstructorref{simplelists.cons}{cons} \coqdocvar{x} .. (\coqconstructorref{simplelists.cons}{cons} \coqdocvar{y} \coqconstructorref{simplelists.nil}{nil}) ..).\coqdoceol
\coqdocemptyline
}\end{coqdoccode}
}

\frame {
  \frametitle{More Lists}
  
  Now we can define a few functions over lists \begin{coqdoccode}
\scriptsize{
    \coqdocemptyline
    \coqdocnoindent
    \coqdockw{Fixpoint} \coqdefinition{simplelists.repeat}{repeat} (\coqdocvar{n} \coqdocvar{count} : \coqinductiveref{Coq.Init.Datatypes.nat}{nat}) \{\coqdockw{struct} \coqdocvar{count}\} : \coqinductiveref{simplelists.natlist}{natlist} := \coqdoceol
    \coqdocindent{1.00em}
    \coqdockw{match} \coqdocvar{count} \coqdockw{with}\coqdoceol
    \coqdocindent{1.00em}
    \ensuremath{|} \coqconstructorref{Coq.Init.Datatypes.O}{O} \ensuremath{\Rightarrow} \coqconstructorref{simplelists.nil}{nil}\coqdoceol
    \coqdocindent{1.00em}
    \ensuremath{|} \coqconstructorref{Coq.Init.Datatypes.S}{S} \coqdocvar{count'} \ensuremath{\Rightarrow} \coqdocvar{n} :: (\coqdefinitionref{simplelists.repeat}{repeat} \coqdocvar{n} \coqdocvar{count'})\coqdoceol
    \coqdocindent{1.00em}
    \coqdockw{end}.\coqdoceol
    \coqdocemptyline}
  \end{coqdoccode}
  The \coqdocvar{length} function calculates the length of a
  list. \begin{coqdoccode}
\scriptsize {
    \coqdocemptyline
    \coqdocnoindent
    \coqdockw{Fixpoint} \coqdefinition{simplelists.length}{length} (\coqdocvar{l}:natlist) \{\coqdockw{struct} \coqdocvar{l}\} : \coqinductiveref{Coq.Init.Datatypes.nat}{nat} := \coqdoceol
    \coqdocindent{1.00em}
    \coqdockw{match} \coqdocvar{l} \coqdockw{with}\coqdoceol
    \coqdocindent{1.00em}
    \ensuremath{|} \coqconstructorref{simplelists.nil}{nil} \ensuremath{\Rightarrow} \coqconstructorref{Coq.Init.Datatypes.O}{O}\coqdoceol
    \coqdocindent{1.00em}
    \ensuremath{|} \coqdocvar{h} :: \coqdocvar{t} \ensuremath{\Rightarrow} \coqconstructorref{Coq.Init.Datatypes.S}{S} (\coqdefinitionref{simplelists.length}{length} \coqdocvar{t})\coqdoceol
    \coqdocindent{1.00em}
    \coqdockw{end}.\coqdoceol
    \coqdocemptyline}
  \end{coqdoccode}
  The \coqdocvar{app} function concatenates two
  lists. \begin{coqdoccode}
\scriptsize{
    \coqdocemptyline
    \coqdocnoindent
    \coqdockw{Fixpoint} \coqdefinition{simplelists.app}{app} (\coqdocvar{l1} \coqdocvar{l2} : \coqinductiveref{simplelists.natlist}{natlist}) \{\coqdockw{struct} \coqdocvar{l1}\} : \coqinductiveref{simplelists.natlist}{natlist} := \coqdoceol
    \coqdocindent{1.00em}
    \coqdockw{match} \coqdocvar{l1} \coqdockw{with}\coqdoceol
    \coqdocindent{1.00em}
    \ensuremath{|} \coqconstructorref{simplelists.nil}{nil}    \ensuremath{\Rightarrow} \coqdocvar{l2}\coqdoceol
    \coqdocindent{1.00em}
    \ensuremath{|} \coqdocvar{h} :: \coqdocvar{t} \ensuremath{\Rightarrow} \coqdocvar{h} :: (\coqdefinitionref{simplelists.app}{app} \coqdocvar{t} \coqdocvar{l2})\coqdoceol
    \coqdocindent{1.00em}
    \coqdockw{end}.\coqdoceol
    \coqdocemptyline
    \coqdocemptyline
    \coqdocnoindent
    \coqdockw{Notation} "x ++ y" := (\coqdefinitionref{simplelists.app}{app} \coqdocvar{x} \coqdocvar{y}) \coqdoceol
    \coqdocindent{10.50em}
    (\coqdocvar{right} \coqdocvar{associativity}, \coqdoctac{at} \coqdocvar{level} 60).\coqdoceol
    \coqdocemptyline}
  \end{coqdoccode}
}
\frame{
  \frametitle{List Theorems}
  One trivial theorem we can show is the indentity of a list
  using 
 
\begin{coqdoccode}
\scriptsize{
    \coqdocnoindent
    \coqdockw{Theorem} \coqlemma{simplelists.nilapp}{nil\_app} : \ensuremath{\forall} \coqdocvar{l}:natlist,\coqdoceol
    \coqdocindent{1.00em}
    [] ++ \coqdocvar{l} = \coqdocvar{l}.\coqdoceol
    \coqdocnoindent
    \coqdockw{Proof}.\coqdoceol
    \coqdocindent{1.50em}
    \coqdoctac{reflexivity}. \coqdockw{Qed}.\coqdoceol}
\end{coqdoccode}
A more complex theorem involving induction

 \begin{coqdoccode}
\scriptsize{
\coqdocnoindent
\coqdockw{Theorem} \coqdocvar{assoc\_app} : \ensuremath{\forall} \coqdocvar{l1} \coqdocvar{l2} \coqdocvar{l3} : \coqdocvar{natlist}, \coqdoceol
\coqdocindent{1.00em}
\coqdocvar{l1} ++ (\coqdocvar{l2} ++ \coqdocvar{l3}) = (\coqdocvar{l1} ++ \coqdocvar{l2}) ++ \coqdocvar{l3}.\coqdoceol
\coqdocnoindent
\coqdockw{Proof}.\coqdoceol
\coqdocindent{1.00em}
\coqdoctac{intros} \coqdocvar{l1} \coqdocvar{l2} \coqdocvar{l3}. \coqdoctac{induction} \coqdocvar{l1} \coqdockw{as} [\ensuremath{|} \coqdocvar{n} \coqdocvar{l1'}].\coqdoceol
\coqdocindent{1.00em}
\coqdocvar{Case} "l1 = nil".\coqdoceol
\coqdocindent{2.00em}
\coqdoctac{reflexivity}.\coqdoceol
\coqdocindent{1.00em}
\coqdocvar{Case} "l1 = cons n l1'".\coqdoceol
\coqdocindent{2.00em}
\coqdoctac{simpl}. \coqdoctac{rewrite} \ensuremath{\rightarrow} \coqdocvar{IHl1'}. \coqdoctac{reflexivity}. \coqdockw{Qed}.\coqdoceol
}
\end{coqdoccode}
}

\frame {
  \frametitle{Resources}
  \begin{itemize}
    \item Coq: http://coq.inria.fr/
    \item Software Foundations http://www.cis.upenn.edu/~bcpierce/sf/
      (from which i heavily borrowed for these slides)
\end{itemize}
}



\end{document}
   