\documentclass{beamer}
\usepackage{beamerthemesplit}
\usepackage[style=authoryear,hyperref=true,citestyle=numeric]{biblatex} 
\usepackage{hyperref}
\usepackage{listings}
\usepackage[all]{xy}
\usepackage[utf8]{inputenc}

% Default fixed font does not support bold face
\DeclareFixedFont{\ttb}{T1}{txtt}{bx}{n}{12} % for bold
\DeclareFixedFont{\ttm}{T1}{txtt}{m}{n}{12}  % for normal

% Custom colors
\usepackage{color}
\definecolor{deepblue}{rgb}{0,0,0.5}
\definecolor{deepred}{rgb}{0.6,0,0}
\definecolor{deepgreen}{rgb}{0,0.5,0}

\useoutertheme{infolines}

\addbibresource{presentation}

% Python style for highlighting
\newcommand\pythonstyle{\lstset{
language=Python,
basicstyle=\tiny\sffamily,
otherkeywords={self},             % Add keywords here
keywordstyle=\tiny\sffamily\color{deepblue},
emph={MyClass,__init__},          % Custom highlighting
emphstyle=\tiny\sffamily\color{deepred},    % Custom highlighting style
stringstyle=\color{deepgreen},
frame=tb,                         % Any extra options here
showstringspaces=false            % 
}}


% Python environment
\lstnewenvironment{python}[1][]
{
\pythonstyle
\lstset{#1}
}
{}

\title{Hadoop Hears a Who}
\author{Dan Colish}
\date{\today}

\begin{document}
\frame{\titlepage}

\section{What is Hadoop}

\frame
{
  \frametitle{Webcrawlers and storage}
  In 2004, the opensource webcrawler project \emph{Nutch} was looking for a
  storage and data processing solution.

  \begin{itemize}
    \item Provide fault tolerance
    \item Scale to storing millions of pages
    \item Avoid read/write bottlenecks
    \item Process large amounts of data on demand
  \end{itemize}

}


\frame
{
  \frametitle{HDFS: Hadoop's Filesystem}

  HDFS is a distributed, fault tolerant userspace filesystem.

  \begin{itemize}
    \item Composed of two core services: the namenode and the datanode
    \item Files are split into blocks
    \item Replicates blocks as they are written
    \item Balances blocks, all nodes share ~= partition of data
    \item Provides partition tolerance against datanode loss
  \end {itemize}

}

\frame
{
  \frametitle{Architecture of HDFS}
  \includegraphics[trim=0 0 0 15mm, clip, width=\linewidth]{hdfsarchitecture.png}
  \cite{hadoopdocs}
}

\frame 
{
  \frametitle{Namenode}
  \begin{itemize}
    \item Controls filesystem metadata and block placement
    \item Executes all non-idempotent file and directory operations
    \item Weak point of typical deployment (SPOF)
    \item Supervises datanode health
  \end{itemize}
}

\frame 
{
  \frametitle{Datanode}
  \begin{itemize}
    \item Serves read and write requests
    \item Executes all block operations
  \end{itemize}  
}


\frame
{
  \frametitle{MapReduce: Hadoop's Data Processing System}
  \begin{displaymath}
    \xymatrix{ 
      *+[Fe:blue]{JobClient} \ar[dr] & & \\
      & *+[F=:magenta]{JobTracker} \ar[dl] \ar[d] \ar[dr] \\
      *+[F]{TaskTracker} \ar[d] & *+[F]{TaskTracker} \ar[d] & *+[F]{TaskTracker} \ar[d] \\
      *+[Fe]{Task} & *+[Fe]{Task} & *+[Fe]{Task}
    }
  \end{displaymath}
}


\frame 
{
  \frametitle{JobTracker}
  \begin{itemize}
    \item Schedules Jobs
    \item Supervises TaskTrackers
  \end{itemize}
}

\frame 
{
  \frametitle{TaskTracker}
  \begin{itemize}
    \item Executes map and reduce tasks
    \item Tasks run in a child jvm
  \end{itemize}
}

\section{Developing MapReduce Jobs}

\frame
{
  \frametitle{The Lifecycle of a MapReduce job}
  \begin{itemize}
    \item Split: input files are split and distributes to mappers
    \item Map: Mappers group raw input to common keys
    \item Shuffle: Distrubuted mapper output is moved to single reducer based on
      keys
    \item Reduce: Groups values are combined by reducer and written to file
      output
  \end{itemize}
}

\begin{frame}[fragile]
  \frametitle{A Sample Job}
  \begin{lstlisting}[language=java,basicstyle=\tiny\sffamily]
    public class Sample extends Configured implements Tool {
      int run(String[] args) {
        ...
      }

      public static class MapClass extends Mapper<Text, Text, Text, Int> {
        @Override
        public void map(Text key, Text value, Mapper.Context context) {
          context.write(key, value);
        }
      }

      public static class Reduce extends Reducer<Text, Text, Text, IntWritable> {
        private IntWritable countWritable = new IntWritable(0);
        @Override
        public void map(Text key, Iterable<Text> values, Mapper.Context context) {
          for (Text _ : values) count++;
          countWritable.set(count);
          context.write(key, countWritable);
        }
      }
    }
  \end{lstlisting}
\end{frame}

\frame
{
  \frametitle{Other MapReduce langauges}
  \begin{itemize}
    \item Hadoop Streaming: Use with shell stdin/stdout
    \item Pig: Semi-structured scripting language, \url{http://pig.apache.org/}
    \item Hive: SQL-like scripting language, \url{http://hive.apache.org/}
    \item Cascalog: Prolog-like clojure meta language, \url{https://github.com/nathanmarz/cascalog}
  \end{itemize}
}

\frame
{
  \frametitle{Submitting and Monitoring Jobs}
  \begin{itemize}
    \item Send job jar and any required files to JobTracker
    \item JobTracker stores job jar and required files in distributed cache
    \item JobTracker schedules execution
    \item TaskTrackers update progress to JobTracker
  \end{itemize}
}

\section{Working with Clusters}

\frame 
{
  \frametitle{Hosted Hadoop}
  \begin{itemize}
    \item Cluster mgmt is time consuming and expensive.
    \item Off load to service providers when possible
    \item Use Amazon Elastic MapReduce (EMR)
    \item Many utility libraries to work with EMR:
      \begin{itemize}
        \item boto: \url{http://docs.pythonboto.org/en/latest/}
        \item mrjob: \url{http://pythonhosted.org/mrjob/}
        \item lemur: \url{https://github.com/TheClimateCorporation/lemur}
      \end{itemize}
  \end{itemize}
}

\begin{frame}[fragile]
  \frametitle{Sample EMR Script}
  \begin{python}
    import datetime, os
    import boto
    from boto.emr.instance_group import InstanceGroup
    from boto.emr.step import InstallPigStep, PigStep

    conn = boto.connect_emr()
    instance_groups = [
        InstanceGroup(1, 'MASTER', 'm1.small', 'SPOT', 'master@0.10', '0.10'),
        InstanceGroup(2, 'CORE', 'm1.small', 'SPOT', 'master@0.10', '0.10'),
        ]

    pig_file = 's3://elasticmapreduce/samples/pig-apache/do-reports2.pig'
    INPUT = 's3://elasticmapreduce/samples/pig-apache/input/'
    OUTPUT = ('s3://org.unencrypted.emr.output/apache_sample/%s' %
              datetime.datetime.utcnow().strftime("%s"))
    pig_args = ['-p', 'INPUT=%s' % INPUT,
                '-p', 'OUTPUT=%s' % OUTPUT]

    pig_step = PigStep('Process Reports', pig_file, pig_args=pig_args)
    steps = [InstallPigStep(), pig_step]

    job_id = conn.run_jobflow(name='sample apache report',
        ec2_keyname=os.getenv("EC2_KEY_NAME"), steps=steps,
        log_uri="s3://org.unencrypted.emr.log/sampleflow_logs", enable_debugging=True,
        ami_version="latest", instance_groups=instance_groups, keep_alive=True)
  \end{python}
\end{frame}

\frame
{
  \frametitle{References}
  \printbibliography
}

\end{document}

% HA setup wtih hdfs
% http://hadoop.apache.org/docs/current/hadoop-yarn/hadoop-yarn-site/HDFSHighAvailabilityWithQJM.html
