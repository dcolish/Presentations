\documentclass{beamer}
\usepackage{beamerthemesplit}
\usepackage[style=authoryear,hyperref=true,citestyle=numeric]{biblatex} 
\usepackage{hyperref}
\usepackage[all]{xy}

\addbibresource{presentation}

\title{Hadoop Hears a Who}
\author{Dan Colish}
\date{\today}


\begin{document}
\frame{\titlepage}

\section{A brief overview of Hadoop}
\subsection{What is Hadoop}

\frame
{
  \frametitle{Webcrawlers and storage}
  In 2004, the opensource webcrawler project \emph{Nutch} was looking for a
  storage and data processing solution.

  \begin{itemize}
    \item Provide fault tolerance
    \item Scale to storing millions of pages
    \item Avoid read/write bottlenecks
    \item Process large amounts of data on demand
  \end{itemize}

}


\frame
{
  \frametitle{HDFS: Hadoop's Filesystem}

  HDFS is a distributed, fault tolerant userspace filesystem.

  \begin{itemize}
    \item Composed of two core services: the namenode and the datanode
    \item Replicates blocks as they are written
    \item Balances blocks, all nodes share ~= partition of data
    \item Provides partition tolerance against datanode loss
  \end {itemize}

}


\frame
{
  \frametitle{Architecture of HDFS}
  \includegraphics[width=\textwidth]{hdfsarchitecture.png}
  \cite{hadoopdocs}
}

\frame
{
  \frametitle{MapReduce: Hadoop's Data Processing System}
  \begin{displaymath}
    \xymatrix{ 
      *+[Fe:blue]{JobClient} \ar[dr] & & \\
      & *+[F=:magenta]{JobTracker} \ar[dl] \ar[d] \ar[dr] & \\
      *+[F]{TaskTracker} & *+[F]{TaskTracker} & *+[F]{TaskTracker}
    }
  \end{displaymath}
}


\subsection{Hadoop Services in Depth}

\frame 
{
  \frametitle{Namenode}
}

\frame 
{
  \frametitle{Datanode}
}

\frame 
{
  \frametitle{JobTracker}
}

\frame 
{
  \frametitle{TaskTracker}
}

\subsection{Developing MapReduce Jobs}

\frame
{
  \frametitle{The Lifecycle of a MapReduce job}

}

\frame
{
  \frametitle{A Sample Job}

}

\frame
{
  \frametitle{Submitting and Monitoring Jobs}
}


\subsection{Developing a MapReduce Job}

\frame
{
  \frametitle{Writing Jobs in Java}
}


\subsection{Running a Cluster}

\frame 
{
  \frametitle{Hosted Hadoop}
  % Cluster mgmt is time consuming and expensive.
  % off load to service providers when possible
}


\frame
{
  \frametitle{References}
  \printbibliography
}

\end{document}

% HA setup wtih hdfs
% http://hadoop.apache.org/docs/current/hadoop-yarn/hadoop-yarn-site/HDFSHighAvailabilityWithQJM.html
