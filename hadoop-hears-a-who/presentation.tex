\documentclass{beamer}
\usepackage{beamerthemesplit}
\usepackage[style=authoryear,hyperref=true,citestyle=numeric]{biblatex} 
\usepackage{hyperref}
\usepackage{listings}
\usepackage[all]{xy}

\useoutertheme{infolines}

\addbibresource{presentation}

\title{Hadoop Hears a Who}
\author{Dan Colish}
\date{\today}

\begin{document}
\frame{\titlepage}

\section{What is Hadoop}

\frame
{
  \frametitle{Webcrawlers and storage}
  In 2004, the opensource webcrawler project \emph{Nutch} was looking for a
  storage and data processing solution.

  \begin{itemize}
    \item Provide fault tolerance
    \item Scale to storing millions of pages
    \item Avoid read/write bottlenecks
    \item Process large amounts of data on demand
  \end{itemize}

}


\frame
{
  \frametitle{HDFS: Hadoop's Filesystem}

  HDFS is a distributed, fault tolerant userspace filesystem.

  \begin{itemize}
    \item Composed of two core services: the namenode and the datanode
    \item Files are split into blocks
    \item Replicates blocks as they are written
    \item Balances blocks, all nodes share ~= partition of data
    \item Provides partition tolerance against datanode loss
  \end {itemize}

}

\frame
{
  \frametitle{Architecture of HDFS}
  \includegraphics[trim=0 0 0 15mm, clip, width=\linewidth]{hdfsarchitecture.png}
  \cite{hadoopdocs}
}

\frame 
{
  \frametitle{Namenode}
  \begin{itemize}
    \item Controls filesystem metadata and block placement
    \item Executes all non-idempotent file and directory operations
    \item Weak point of typical deployment (SPOF)
    \item Supervises datanode health
  \end{itemize}
}

\frame 
{
  \frametitle{Datanode}
  \begin{itemize}
    \item Serves read and write requests
    \item Executes all block operations
  \end{itemize}  
}


\frame
{
  \frametitle{MapReduce: Hadoop's Data Processing System}
  \begin{displaymath}
    \xymatrix{ 
      *+[Fe:blue]{JobClient} \ar[dr] & & \\
      & *+[F=:magenta]{JobTracker} \ar[dl] \ar[d] \ar[dr] \\
      *+[F]{TaskTracker} \ar[d] & *+[F]{TaskTracker} \ar[d] & *+[F]{TaskTracker} \ar[d] \\
      *+[Fe]{Task} & *+[Fe]{Task} & *+[Fe]{Task}
    }
  \end{displaymath}
}


\frame 
{
  \frametitle{JobTracker}
  \begin{itemize}
    \item Schedules Jobs
    \item Supervises TaskTrackers
  \end{itemize}
}

\frame 
{
  \frametitle{TaskTracker}
  \begin{itemize}
    \item Executes map and reduce tasks
    \item Tasks run in a child jvm
  \end{itemize}
}

\section{Developing MapReduce Jobs}

\frame
{
  \frametitle{The Lifecycle of a MapReduce job}
  \begin{itemize}
    \item Split: input files are split and distributes to mappers
    \item Map: Mappers group raw input to common keys
    \item Shuffle: Distrubuted mapper output is moved to single reducer based on
      keys
    \item Reduce: Groups values are combined by reducer and written to file
      output
  \end{itemize}
}

\begin{frame}[fragile]
  \frametitle{A Sample Job}
  \begin{lstlisting}[language=java,basicstyle=\tiny\sffamily]
    public class Sample extends Configured implements Tool {
      public static class MapClass extends Mapper<Text, Text, Text, Int> {
        @Override
        public void map(Text key, Text value, Mapper.Context context) {
          context.write(key, value);
        }
      }

      public static class Reduce extends Reducer<Text, Text, Text, IntWritable> {
        private IntWritable countWritable = new IntWritable(0);
        @Override
        public void map(Text key, Iterable<Text> values, Mapper.Context context) {
          for (Text _ : values) count++;
          countWritable.set(count);
          context.write(key, countWritable);
        }
      }
    }
  \end{lstlisting}
\end{frame}


\frame
{
  \frametitle{Submitting and Monitoring Jobs}
  % \begin{itemize}
  % \end{itemize}

}

\frame
{
  \frametitle{Writing Jobs in Java}
  % \begin{itemize}
  % \end{itemize}
}


\section{Running a Cluster}

\frame 
{
  \frametitle{Hosted Hadoop}
  \begin{itemize}
    \item Cluster mgmt is time consuming and expensive.
    \item Off load to service providers when possible
  \end{itemize}
}


\frame
{
  \frametitle{References}
  \printbibliography
}

\end{document}

% HA setup wtih hdfs
% http://hadoop.apache.org/docs/current/hadoop-yarn/hadoop-yarn-site/HDFSHighAvailabilityWithQJM.html
